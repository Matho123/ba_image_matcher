
\chapter*{Zusammenfassung}
\label{sec:Zusammenfassung}
In dieser Arbeit wird der Entwurf eines Bilderkennungssystems beschrieben. 
Das System ist in der Lage  zu erkennen, ob es sich bei einem neuen Bild um ein Duplikat handelt, das bereits in einer Datenbank mit registrierten Bildern vorhanden ist.
Um Duplikate zu erkennen, deren Motiv im Vergleich zum Original modifiziert, aber immer noch als Kopie erkennbar ist, wird ein feature-based Bilderkennungs-Algorithmus eingesetzt. 
Innerhalb der Arbeit wurden SIFT, ORB und BRISK für das Erkennen von duplizierten T-Shirt Designs der Spread Group miteinander verglichen.
Für den Vergleich wurden die Algorithmen in verschiedenen Szenarien getestet.
