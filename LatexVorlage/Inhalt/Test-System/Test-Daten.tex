\section{Erstellung der Suchbilder}
\label{sec:TestDaten}
F�r die Erstellung der Suchbilder wurden f�r jedes Szenario zuf�llige Bilder aus dem Datenbanksatz genommen.
Zum Erstellen der Variationen wurden f�r fast alle Szenarien ein Skript verwendet.
Das �ndern der Aufl�sung und das spiegeln an den Hauptachsen sind einfache Transformationen, die sich leicht automatisieren lassen.

Da alle Bilder aus dem Datenbankset einen transparenten Hintergrund haben, lassen sich diese problemlos auf die Dimensionen des Motivs zuschneiden.
Dadurch k�nnen die Motive verschoben und rotiert werden, ohne das sich der Hintergrund mit bewegt.
Auch das �ndern der Hintergrundfarbe und das Platzieren des Motivs auf einen gr��eren Bild sind bei Bildern mit transparenten Hintergrund unkompliziert automatisierbar.
Einzig das �ndern der F�rbung von Motiven l�sst sich nicht einfach in einem Skript realisieren und wurde daher von Hand gemacht.

Das Skript ist in Go \cite{go} geschrieben und alle Bildtransformationen lie�en sich �ber die Standartbibliotheken implementieren.
F�r die Suchsets wurden pro Szenario 100 bis 150 Variationen erstellt.
 