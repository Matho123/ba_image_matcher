\section{Test-Szenarien}
\label{sec:TestSzenarien}
Aus der Analyse hat sich ergeben, dass ein Bild auf unterschiedliche Weisen ver�ndert werden und trotzdem als Duplikat gelten kann.
Daher sollen Test-Szenarien definiert werden, in denen die Algorithmen getestet werden.
Jedes Szenario bildet dabei eine Modifikation ab, die auf ein Bild angewendet werden kann, sodass es im Sinne der Arbeit immer noch als Duplikat gilt.

In den folgenden Szenarien soll getestet werden:
\begin{itemize}
\item \textbf{identische Kopie}
Im Suchset sind pixelgenaue Kopien aus dem Datenbank-Set enthalten.
Dieses Szenario stellt den Trivialfall dar, bei eine Balancierte-Genauigkeit von 100\% und ein �hnlichkeits-Score von 1 bei Duplikaten erwartet wird. 
Eine niedrigere Balancierte-Genauigkeit oder ein niedrigerer �hnlichkeits-Score lassen darauf schlie�en, dass der getestete Algorithmus nicht richtig implementiert und die Bild�hnlichkeits-Formel fehlerhaft ist.
\item \textbf{Ge�nderte Aufl�sung}
Es werden f�r die duplizierten Bilder niedriger aufgel�ste Varianten erstellt. Dabei werden die 3 Skalierungsfaktoren 2, 4, 10 und angewendet. F�r jeden Skalierungsfaktor wird eine Variation erstellt, deren Seitenl�ngen um den Skalierungsfaktor gek�rzt werden.
\item \textbf{Verschobenes Motiv}
Das Motiv des Duplikats ist im Vergleich zum Original verschoben. 
\item \textbf{Rotiertes Motiv}
Die Duplikate sind in diesem Fall im Vergleich zum Original rotiert. Es werden die Rotationswinkel 5�, 10�, 45�, 90� und 180� verwendet. 
F�r jedes duplizierte Bild wird f�r jeden Winkel eine Variation erstellt.
\item \textbf{Gespiegeltes Motiv}
F�r jedes duplizierte Bild existiert in diesem Szenario zwei Varianten. Die eine Variante ist an der X-Achse gespiegelt und die andere Variante ist an der Y-Achse gespiegelt.
\item \textbf{Andere Hintergrundfarbe}
Die Duplikate haben in diesem Szenario im Vergleich zum Originalbild eine andere Hintergrundfarbe.
Die neue Hintergrundfarbe muss gen�gend Kontrast zum Motiv haben, sodass das Motiv noch erkennbar ist. 
\item \textbf{Motiv in anderer Farbe}
In diesem Szenario haben die Duplikate eine anders gef�rbtes Motiv. 
Daf�r werden Originalbilder mit flachen Farbstil verwendet. 
Diese lassen sich leichter neu einf�rben.
\item \textbf{Originalbild als teil eines gr��eren Bildes}
F�r Duplikate in diesem Szenario wird das Originalbild auf einem gr��eren Bild platziert. 
\item \textbf{Misch-Szenario} 
Alle anderen Szenarien kommen in diesem Szenario vor. 
Ziel ist es eine realistische Anwendungssituation darzustellen. 
Dazu werden auch Duplikate mit mehreren Modifikationen erstellt.
Beispielsweise k�nnen Duplikate, die sowohl rotiert als auch eine niedriger Aufl�sung haben, vorkommen.
\end{itemize}

F�r alle Szenarien werden die, in den Grundlagen definierten Metriken (Abschnitt \ref{sec:TestMetriken}), festgehalten. 
Dadurch kann jedes Szenario einzeln ausgewertet werden und Aufschluss �ber die individuellen St�rken und Schw�chen der getesteten Algorithmen geben. 

In den Tests der feature-based Algorithmen soll in jedem Szenario mit verschiedenen Mindestwerten f�r den Bild�hnlichkeits-Score experimentiert werden. Ziel ist es f�r jeden Algorithmus einen Mindestwert zu finden, der f�r alle Szenarien eine m�glichst hohen Recall und eine gute Balancierte-Genauigkeit erreicht.

Zudem werden im Misch-Szenario Laufzeit und Speicherkosten der einzelnen Algorithmen dokumentiert. Da nur in diesem Szenario, die Geschwindigkeit eine Rolle spielt, werden hier der Brute-Force und FLANN-Based Matcher verglichen. 
Da es bei allen anderen Szenarien um die Genauigkeit geht, wird dort nur der Brute-Force Matcher eingesetzt.
