\section{Test-Szenarien}
\label{sec:TestSzenarien}
Aus der Analyse hat sich ergeben, dass ein Bild auf unterschiedliche Weisen ver�ndert werden und trotzdem als Duplikat gelten kann.
Daher sollen Test-Szenarien definiert werden, in denen die Algorithmen getestet werden.
Jedes Szenario bildet dabei eine Modifikation ab, die auf ein Bild angewendet werden kann, sodass es im Sinne der Arbeit immer noch als Duplikat gilt.

In den folgenden Szenarien soll getestet werden:
\begin{itemize}
\item \textbf{1-zu-1-Kopien}
Im Suchset sind pixelgenaue Kopien aus dem Datenbank-Set enthalten.
Dieses Szenario stellt den Trivialfall dar und es wird hierbei eine Balancierte-Genauigkeit von 100\% und ein �hnlichkeits-Score von 1 bei Duplikaten erwartet. 
Eine niedrigere Balancierte-Genauigkeit oder ein niedrigerer �hnlichkeits-Score lassen darauf schlie�en, dass der getestete Algorithmus nicht richtig implementiert und die Bild�hnlichkeits-Formel fehlerhaft ist.
\item \textbf{Ge�nderte Aufl�sung}
\item \textbf{Verschobenes Motiv}
\item \textbf{Rotiertes Motiv}
\item \textbf{Gespiegeltes Motiv}
\item \textbf{Andere Hintergrundfarbe}
\item \textbf{Motiv in anderer Farbe}
\item \textbf{Originalbild als teil eines gr��eren Bildes}
\item \textbf{Misch-Szenario} 
Alle anderen Szenarien kommen in diesem Szenario vor. 
Ziel ist es eine realistische Anwendungssituation darzustellen. 
Dazu werden auch Duplikate mit mehreren Modifikationen erstellt.
Beispielsweise k�nnen Duplikate, die sowohl rotiert als auch eine niedriger Aufl�sung haben, vorkommen.
\end{itemize}

F�r alle Szenarien werden die, in den Grundlagen definierten Metriken (Abschnitt \ref{sec:TestMetriken}), festgehalten. 
Dadurch jedes Szenario kann einzeln ausgewertet werden und Aufschluss �ber die individuellen St�rken und Schw�chen der getesteten Algorithmen liefern. 

In den Tests der feature-based Algorithmen soll in jedem Szenario mit verschiedenen Mindestwerten f�r den Bild�hnlichkeits-Score experimentiert werden. Ziel ist es f�r jeden Algorithmus einen Mindestwert zu finden, der f�r alle Szenarien eine m�glichst gute Balancierte-Genauigkeit erreicht.

Zudem werden im Misch-Szenario Laufzeit und Speicherkosten der einzelnen Algorithmen dokumentiert.
