
\chapter{Einleitung}
\label{sec:Einleitung}

\section{Motivation}
\label{sec:Motivation}
Bei Spreadshirt werden t�glich tausende Designs hochgeladen. 
Viele davon in der Partner-Area, wo Designer ihre Bilder zum Verkauf anbieten k�nnen. 
Dabei muss seitens Spreadshirt sichergestellt werden, dass die hochgeladenen Designs nicht verfassungsfeindlich sind und auch nicht gegen das Urheberrecht oder Spreadshirts eigene Richtlinien versto�en. 
Die manuelle �berpr�fung von so vielen Designs ist sehr aufw�ndig und h�ufig kommt es vor, dass die gleichen Designs mehrmals hochgeladen werden. 
Eine automatische Sperrung, von bereits verbotenen Designs, die erneut hochgeladen werden, kann den �berpr�fungsprozess entlasten.
Daher ist ein System notwendig, dass bei neu hochgeladenen Designs Duplikate in der Datenbank der verbotenen Designs erkennen kann.

\section{Aufgabenstellung}
\label{sec:Aufgabenstellung}
Ziel dieser Bachelorarbeit ist die Entwicklung eines Systems, dass in der Lage ist, bei neuen Bildern zu erkennen, ob diese bereits in einer Datenbank mit bereits gespeicherten Bildern auftauchen. 
Ein neues Bild soll auch dann als Duplikat erkannt werden, wenn der Bildinhalt im Vergleich zum Original transformiert, also rotiert, skaliert, verschoben oder gespiegelt wurde. 
F�lle in denen das Bild anders gef�rbt ist oder als Teil eines gr��eren Bildes auftaucht, sollen ebenfalls ber�cksichtigt werden.

Zur Implementierung des Systems soll ein "feature-based" (dt. merkmalbasierter) Algorithmus verwendet werden. 
Es gibt eine Vielzahl an merkmalbasierten Algorithmen, die jeweils ihre eigenen St�rken und Schw�chen haben. 
Eine Auswahl dieser Algorithmen sollen innerhalb der Arbeit f�r den Anwendungsfall bei Spreadshirt getestet und miteinander verglichen werden.

\section{Erfolgs- und Qualit�tskriterien}
\label{sec:ErfolgsUndQualit�tskriterien}
Die G�te des Bilderkennungssystems soll anhand von Testdatens�tzen ermittelt werden. 
Die Testdatens�tze sind dabei in zwei Teils�tzen unterteilt. 
Der erste Teilsatz an Bildern stellt die Menge an gespeicherten Datenbankbildern dar. 
Der zweite Teilsatz enth�lt eine Untermenge an Duplikaten aus dem Datenbanksatz und eine Menge an neuen Bildern, die nicht im Datenbanksatz auftauchen. Getestet wird in verschiedenen Szenarien, die die Robustheit des Systems gegen�ber bestimmter Sonderf�lle testen soll. Je nach Szenario sind die Duplikate auf unterschiedliche Weise im Verglich zum Original ver�ndert. Als Vergleich dient der pHash-Algorithmus, der momentan bei Spreadshirt zur Duplikatensuche verwendet wird.

Die verwendeten Metriken werden in den Grundlagen erkl�rt.
Am wichtigsten ist dabei ein hoher Recall, sodass m�glichst viele Duplikate durch das System abgefangen werden. 
Da automatisch gesperrte Designs nochmal manuell gepr�ft werden, f�llt eine niedrigere Spezifizit�t bei der Auswertung nicht so sehr ins Gewicht. Laufzeit- und Speicherkosten sollen ebenfalls innerhalb der Arbeit abgesch�tzt werden. 

Um f�r die Verwendung bei Spreadshirt in Frage zu kommen, muss das neue System zuverl�ssiger sein, als die momentan eingesetzte pHash-Implementaion. 
Daf�r wird ein h�herer Recall angestrebt. Um sicherzustellen, dass Spezifizit�t nicht zu sehr absinkt, wird auf eine h�here oder zumindest gleichbleibende Balancierte-Genauigkeit abgezielt.



