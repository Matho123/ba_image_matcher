
\chapter{Einleitung}
\label{sec:Einleitung}

\section{Motivation}
\label{sec:Motivation}
Die Spread Group bietet Nutzern �ber ihre Plattformen die M�glichkeit personalisierte Produkte zu kaufen. 
Dabei k�nnen Nutzer auch eigene Designs hochladen, um diese beispielsweise auf selbst erstellten Produkten zu verkaufen. 
T�glich werden tausende solcher selbst erstellter Designs hochgeladen.
Dabei muss die Spread Group sicherstellen, dass die hochgeladenen Designs nicht gegen Gesetze oder die unternehmensinternen Richtlinien versto�en. 
Die manuelle �berpr�fung von so vielen Designs ist sehr aufw�ndig und h�ufig kommt es vor, dass die gleichen Designs mehrmals hochgeladen werden. 
Eine automatische Sperrung von bereits abgelehnten Designs, die erneut hochgeladen werden, kann den �berpr�fungsprozess entlasten.
Daher ist ein System notwendig, dass erkennen kann, ob ein neu hochgeladenes Design ein bereits abgelehntes Design dupliziert.

\section{Aufgabenstellung}
\label{sec:Aufgabenstellung}
Ziel dieser Bachelorarbeit ist die Entwicklung eines Systems zur Identifikation von Duplikaten in einer Datenbank. 
Das System soll dazu in der Lage sein, neue Designs mit Bildern in der Datenbank zu vergleichen und festzustellen, ob diese identisch oder �hnlich sind. 
Ein neues Bild soll auch dann als Duplikat erkannt werden, wenn der Bildinhalt im Vergleich zum Original transformiert, also rotiert, skaliert, verschoben oder gespiegelt wurde. 
F�lle in denen das Bild anders gef�rbt ist oder als Teil eines gr��eren Bildes auftaucht, sollen ebenfalls ber�cksichtigt werden.

Zur Implementierung des Systems soll ein feature-based (dt. merkmalbasierter) Algorithmus verwendet werden. 
Es gibt eine Vielzahl an feature-based Algorithmen, die jeweils ihre eigenen St�rken und Schw�chen haben. 
Eine Auswahl dieser Algorithmen soll innerhalb der Arbeit f�r den Einsatz in der Bildakkreditierung bei Spread Group getestet und miteinander verglichen werden.

\section{Erfolgs- und Qualit�tskriterien}
\label{sec:ErfolgsUndQualit�tskriterien}
Die G�te des Bilderkennungssystems soll anhand von einem Testdatensatz ermittelt werden. 
Der Testdatensatz ist dabei in zwei Teils�tze unterteilt. 
Der erste Teilsatz an Bildern stellt die Menge an gespeicherten Datenbankbildern dar. 
Der zweite Teilsatz enth�lt eine Untermenge an Duplikaten aus dem Datenbanksatz. 
Au�erdem sind auch eine Menge an neuen Bildern enthalten, die nicht im Datenbanksatz auftauchen. 
Getestet wird in verschiedenen Szenarien, die die Robustheit des Systems gegen�ber bestimmter Sonderf�lle testen sollen. 
Je nach Szenario sind die Duplikate auf unterschiedliche Weise im Vergleich zum Original ver�ndert. 
Als Vergleich dient der pHash-Algorithmus, der momentan in der Sprad Group zur Duplikatensuche verwendet wird.

Die verwendeten Metriken werden in den Grundlagen \ref{sec:TestMetriken} erkl�rt.
Am wichtigsten ist dabei ein hoher Recall, sodass m�glichst viele Duplikate durch das System abgefangen werden. 
Da automatisch gesperrte Designs nochmal manuell gepr�ft werden, f�llt eine niedrigere Spezifizit�t bei der Auswertung nicht so sehr ins Gewicht. Laufzeit- und Speicherkosten sollen ebenfalls innerhalb der Arbeit abgesch�tzt werden. 

Um f�r die Verwendung bei der Spread Group in Frage zu kommen, muss das neue System Duplikate zuverl�ssiger erkennen k�nnen, als die momentan eingesetzte pHash-Implementation. 
Daf�r wird ein h�herer Recall angestrebt. Um sicherzustellen, dass die Spezifizit�t nicht zu sehr absinkt, wird auf eine h�here oder zumindest gleichbleibende Balancierte-Genauigkeit abgezielt.



