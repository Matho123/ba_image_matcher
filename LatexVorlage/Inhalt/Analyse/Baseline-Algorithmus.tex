\section{Baseline-Algorithmus (pHash)}
\label{sec:pHash}
Spreadshirt nutzt momentan einen Perceptual-Hashing-Algorithmus (kurz pHash) um neu hochgeladene Bilder zu erkennen, die bereits abgelehnt wurden. Die Idee von pHash ist, dass f�r �hnliche Bilder ein �hnlicher pHash-Wert ermittelt wird. Die pHash Implementierung bei Spreadshirt st�tzt sich auf den Blockeintrag \cite{pHash} und l�sst sich grob in drei Phasen unterteilen.

\paragraph{Phase 1: Vorverarbeitungsphase}
Zuerst kommt das Bild in eine Vorverarbeitungsphase. Falls das Bild einen transparenten Hintergrund hat, wird es so zugeschnitten, dass der Bildrand direkt am Rand des Motivs liegt. Danach wird das Bild auf eine Aufl�sung von 32x32 herunter skaliert und in ein Graustufenbild umgewandelt. 

\paragraph{Phase 2: Umwandlung in den Frequenzraum} 
In dieser Phase wird das Bild vom Ortsraum in den Frequenzraum �berf�hrt. 
Dazu wird die diskrete Kosinustransformation (kurz DCT) genutzt. 
Es entsteht eine 32x32 DCT-Matrix bestehend aus Koeffizienten, mit denen, in Kombination mit der Kosinusfunktion, das Originalbild rekonstruiert werden kann. 

F�r pHash wird der obere linke 8x8 Bereich der DCT-Matrix genutzt. Dieser Bereich enth�lt die Niederfrequenzkomponenten und wird als reduzierte DCT-Matrix bezeichnet.  
Die Niederfrequenzkomponenten sind f�r pHash interessant, weil sie die groben Strukturen und Muster des Bildes beschreiben. Zwei Bilder mit �hnlichen Strukturen und Mustern werden �hnliche Niederfrequenzkomponenten haben und damit zu �hnlichen Hash-Werten f�hren. 

\paragraph{Phase 3: Hash-Berechnung}
F�r die Hash-Berechnung wird der Median in der reduzierten DCT-Matrix ermittelt. Die reduzierte DCT-Matrix der Gr��e 8x8 beinhaltet insgesamt 64 Werte. F�r jeden Wert wird gepr�ft, ob er �ber oder unter dem Median liegt. 
Wenn der Wert gr��er ist, wird eine 1 und ansonsten eine 0 an den Hash-Wert angeh�ngt. Damit ergibt sich ein 64-bit Hash-Wert f�r ein Bild.

\paragraph{pHash-Werte vergleichen}
Zum vergleichen von pHash-Werten, wird die Hamming-Distanz (Siehe Abschnitt ~\ref{sec:DeskriptorenVergleichen}) genutzt.
In der Spread Group gilt ein Bild als Duplikat, wenn beim Hash-Vergleich mit einem Bild aus der Datenbank eine Hamming-Distanz kleiner oder gleich 4 ermittelt wird.

