\section{Vorgehen}
\label{sec:Vorgehen}

Aus der Betrachtung der Problemstellung ergeben sich die folgenden Teilaufgaben:

\begin{enumerate}
\item Definition einer Formel zur Bestimmung der �hnlichkeit von zwei Bildern (Kapitel \ref{sec:Formel})
\item Evaluation des Baseline-Algorithmus und der merkmalbasierten Algorithmen
\begin{enumerate}
\item Definition von Test-Szenarien in denen die Algorithmen verglichen werden (Abschnitt \ref{sec:TestSzenarien})
\item Erstellung von Suchbildern f�r die Test-Szenarien (Abschnitt \ref{sec:TestDaten})
\item Entwicklung eines Test-Systems, das die Algorithmen durch die Test-Szenarien laufen l�sst und Statistiken festh�lt (Abschnitt \ref{sec:TestEntwurf} und \ref{sec:TestUmsetzung})
\item Auswertung der Tests zur Evaluation des Baseline-Algorithmus und zur Bestimmung welcher Feature-Based Algorithmus am besten f�r das neue Bilderkennungssystem geeignet ist (Kapitel \ref{sec:VersucheUndAuswertung})
\end{enumerate} 
\item Entwicklung und prototypische Implementierung eines neuen Bilderkennungssystems (Abschnitt \ref{sec:BSGedanken} und \ref{sec:BSEntwurf})
\item Vergleich des neuen Bilderkennungssystems mit dem Baseline-Algorithmus (Abschnitt \ref{sec:BSAuswertung})
\end{enumerate}