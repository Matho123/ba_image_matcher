\section{Duplikate}
\label{sec:Duplikate}

\paragraph{Definition}
Im Kontext dieser Arbeit sind mit Duplikaten Bilder gemeint, die den gleichen Inhalt wie ein bereits bekanntes Bild haben. 
Der Inhalt ist in diesem Fall im optischen und nicht im semantischen Sinne gemeint. Ein Duplikat muss optische gleiche, oder zumindest sehr �hnliche Merkmale im Vergleich zum Original aufweisen. 
Wenn zwei Bilder ein Haus zeigen, und somit im semantischen Sinne den gleichen Inhalt haben, ist das erstmal nicht interessant. 
Nur wenn die H�user �hnliche oder gleiche Merkmale aufweisen, indem sie zum Beispiel den gleichen Aufbau haben, kommt eine Klassifizierung als Duplikat in Frage.

Da der Vergleich von Merkmalen hierbei eine zentrale Rolle spielt, muss ein Bild keine pixelgenaue Kopie sein um als Duplikat zu gelten.
Ein Bild kann auch dann ein Duplikat sein, wenn sein Inhalt im Vergleich zum Original rotiert, skaliert, verschoben oder gespiegelt wurde.
Bilder mit anderer Hintergrundfarbe und anders gef�rbten Motiv k�nnen ebenfalls als Duplikat gelten. 
Wichtig ist, die Form de Merkmale.
Auch wenn das Motiv des Originalbilds als Teil des Motivs, des neuen Bilds auftaucht, soll es im Sinne der Arbeit als Duplikat erkannt werden.

Diese sehr weit gefasste Definition von Duplikaten ist n�tig, damit die Bildakkreditierung bei Spreadshirt nicht so einfach �berlistet werden kann. 
Der Nutzer soll nicht in der Lage sein, auf Spreadshirts Plattform ein verbotenes Bild hochzuladen, wenn er es �ber die genannten Methoden ver�ndert. 
Ein Bild wird aufgrund seines Inhalts verboten. Solange der verbotene Inhalt erkennbar ist, spielt es keine Rolle, wie er transformiert wurde.

\paragraph{Duplikate mit Merkmalen erkennen}
Die in den Grundlagen in Abschnitt ~\ref{sec:Grundlagen} vorgestellten merkmalbasierten Algorithmen liefern Deskriptoren, mit denen bei Deskriptoren von einem anderen Bild nach �bereinstimmungen, auch Matches genannt, gesucht werden k�nnen. 

Die Menge an gefundenen Matches an sich gibt allerdings noch keinen Aufschluss dar�ber, ob die beiden Bilder tats�chlich �bereinstimmen. 
Es braucht eine Formel, die anhand der Matches eine Aussage dar�ber treffen kann, wie �hnlich sich zwei Bilder sind.
Bei meiner Recherche habe ich keine konkrete Formel gefunden, die das bewerkstelligt.
Daher muss eine eigene Formel entwickelt werden.