
\chapter{Grundlagen}
\label{sec:Grundlagen}

\section{Scale Invariant Feature Transform}
\label{sift}

\section{Test-Metriken}
\label{sec:TestMetriken}
Das Bilderkennungssystem soll die Bilder in zwei Klassen einordnet: Duplikate und nicht-Duplikate/neue Bilder. 
Da es in diesem Fall bei der Klassifizierung nur zwei m�gliche Klassen gibt, kann man das Bilderkennungssystem als bin�ren Klassifikator bezeichnen. 

Die Performance von bin�ren Klassifikatoren kann durch die Werte einer Wahrheitsmatrix quantifiziert werden. 
Die Wahrheitsmatrix gibt dabei an, wie viele richtige und falsche Entscheidungen das System bei der Klassifikation getroffen hat. \cite[S. 170]{classification}

\begin{tabular}[h]{l | l | l}
&\multicolumn{2}{c}{Wahre Klassifikation} \\
System Klassifikation: &Duplikat&nicht-Duplikat \\
\hline
Duplikat&Anzahl wahre Duplikate (TP) &Anzahl falsche Duplikate (FP) \\
\hline
nicht-Duplikat&Anzahl falsche nicht-Duplikate (FN) &Anzahl wahre nicht-Duplikate (TN)
\end{tabular}

Aus den Werten der Wahrheitsmatrix lassen sich weitere Metriken ableiten, die f�r die Bewertung eines bin�ren Klassifikators n�tzlich sein k�nnen. Interessant f�r diese Arbeit sind Recall, Spezifizit�t und Balancierte-Genauigkeit.

Der Recall, oder auch true positive rate, gibt das Verh�ltnis zwischen den Duplikaten, die das System korrekt klassifiziert hat, und allen Duplikaten, die sich in dem Suchdatensatz befinden, an. \cite[S. 172]{classification}

$$Recall = \frac{ TP }{ TP + FN }$$

Die Spezifizit�t, oder auch true negative rate, gibt das Verh�ltnis zwischen den nicht-Duplikaten, die das System korrekt klassifiziert hat, und allen nicht-Duplikaten, die sich im Suchdatensatz befinden an. \cite[S. 172]{classification}

Die Accuracy (dt. Genauigkeit), gibt das Verh�ltnis zwischen den Bildern, die das System korrekt klassifiziert hat, und allen Bildern im Suchdatensatz an.
Die Genauigkeit spiegelt die F�higkeit des Systems Bilder richtig zu Klassifizieren wieder.\cite[S. 171]{classification}

Allerdings ist die Genauigkeit kein zuverl�ssiger Wert, wenn man mit unausgeglichenen Datens�tzen arbeitet. 
Ein Datensatz gilt dann als unausgeglichen, wenn einer Klassifikation mehr Elemente angeh�ren, als der anderen Klassifikation. \cite[S. 171]{classification}
In den Testdatens�tzen, die f�r diese Arbeit verwendet werden, befinden sich mehr nicht-Duplikate als Duplikate.
Dieses Ungleichgewicht kann die Genauigkeit stark beeinflussen. So kann zum Beispiel eine gute Spezifizit�t einen schlechten Recall ausgleichen, wenn der Datensatz zum gr��ten Teil aus nicht-Duplikaten besteht.

Daher kommt bei unbalancierten Datens�tzen die balanced-accuracy (dt. Balancierte-Genauigkeit) zum Einsatz. \cite[S. 175]{classification} 
Da sie den Durchschnitt aus Recall und Spezifizit�t bildet, werden Unterschiede zwischen den beiden Werten ausgeglichen. Dadurch ist die Balancierte-Genauigkeit robust gegen�ber unausgeglichenen Datens�tzen.