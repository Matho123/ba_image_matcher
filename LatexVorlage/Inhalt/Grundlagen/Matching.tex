\section{Match-Suche mit Deskriptoren}
\label{sec:Matching}

Im Abschnitt  ~\ref{sec:m-based} wurden die Merkmalbasierten-Algorithmen erkl�rt, mit denen Deskriptoren f�r Merkmale in einem Bild generiert werden k�nnen.
In diesem Abschnitt soll es darum gehen, wie diese Deskriptoren genutzt werden k�nnen, um die Merkmale aus zwei Bildern zu vergleichen und �bereinstimmungen zu finden. �bereinstimmungen werden auch als Match bezeichnet.

\subsection{Deskriptoren Vergleichen}
\label{sec:DeskriptorenVergleichen}

Um zwei Deskriptoren zu vergleichen, wird die Distanz zwischen den beiden berechnet. 
Die Distanz beschreibt in diesem Fall wie �hnlich sich Deskriptoren sind. 
Je kleiner die Distanz desto �hnlicher die Deskriptoren.
Abh�ngig davon welche Art Deskriptor verglichen wird, wird ein anderer Distanzma�stab verwendet.

\paragraph{Euklidische-Distanz}
Bei Deskriptoren, die aus dem SIFT-Algorithmus resultieren, wird die Euklidische Distanz genutzt. \cite[S. 104]{sift2004} Die Deskriptoren sind hierbei eine Folge von Gradient-Werten. 
Stelle f�r Stelle wird die Euklidische Distanz f�r alle Gradient-Werte der Deskriptoren berechnet und aufsummiert. 

$$distance(desc1, desc2) := \sum_{ i = 1 }^{ 128 }{ \sqrt{ (desc1[i] - desc2[i])^2 } }$$

SIFT-Deskriptoren bestehen in der Regel aus 128 Gradient-Werten.

\paragraph{Hamming-Distanz}
Die Hamming-Distanz kann verwendet werden, um Bitfolgen zu vergleichen. 
Dazu werden beide Bitfolgen Stelle f�r Stelle verglichen. 
F�r jede Stelle, die nicht �bereinstimmt, steigt die Distanz um 1.

$$distance(desc1, desc2) := \vert \{ i \in \{1,..., 256\} \mid desc1_{i} \neq desc2_{i}\} \vert$$

Die Hamming-Distanz eignet sich um Deskriptoren, die durch den ORB- oder BRISK-Algorithmus errechnet werden, zu vergleichen. \cite[S.  2569]{orb2011} 
Bei diesen handelt es sich um Bitfolgen mit 256 Stellen bei ORB und 512 Stellen bei BRISK. Auch bei dem Vergleich von pHash-Werten kommt die Hamming-Distanz zum Einsatz. 

Durch bitweises XOR kann die Hamming-Distanz deutlich effizienter berechnet werden, als die Euklidische Distanz. 
Bin�r-Deskriptoren haben dadurch beim Vergleichen einen Geschwindigkeitsvorteil.

\subsection{Matching Strategien}
\label{sec:MatchingStrategien}

\paragraph{Brute-Force Matcher}
\label{sec:BFM}
Um �bereinstimmungen von Merkmalen in zwei Bildern zu finden, werden alle Deskriptoren des ersten Bilds mit allen Deskriptoren des zweiten Bilds verglichen. \cite{opencv}

Der Vorteil dieser simplen Herangehensweise ist, dass garantiert die besten Matches gefunden werden. 
Allerdings ist diese Strategie mit $\mathcal{O}(n^2)$ auch sehr ineffizient, wodurch sie gerade bei Datensets mit einer gro�en Menge an Deskriptoren ins Gewicht f�llt. 

\paragraph{FLANN-Based Matcher}
FLANN ist eine Bibliothek f�r fast approximate nearest neighbor search. 
Abh�ngig vom verwendeten Datensatz w�hlt die Bibliothek automatisch aus einer Reihe von Algorithmen, die f�r approximate nearest neighbor Suche optimiert sind, den besten aus. F�r gr��ere Datensets arbeitet FLANN schneller als der Brute-Force Matcher. Allerdings kann durch die approximate nearest neighbor search nicht garantiert werden, dass FLANN die besten Matches findet. \cite{opencv}


\subsection{Filterung der Matching-Ergebnisse}
\label{sec:RatioTest}
F�r jeden Deskriptor aus dem ersten Bild wird der  erst- und zweitbeste Match aus dem zweiten Bild gespeichert. Ein Match besteht dabei aus den beiden Deskriptoren, die verglichen werden, und die resultierende Distanz (Siehe Abschnit ~\ref{sec:DeskriptorenVergleichen}). Daraus ergibt sich eine 2xN Matrix, wobei N die Anzahl der Deskriptoren im ersten Bild darstellt.

Es k�nnen Deskriptoren vorkommen, f�r die keine zuverl�ssiger Match gefunden werden kann. Das kann unter anderen passieren, wenn Schl�sselpunkte an R�ndern von Objekten oder wirren Bereichen im Bild gefunden werden. 
Weil mit solchen Deskriptoren keine gute Aussage �ber die �hnlichkeit von zwei Bildern getroffen werden kann, m�ssen diese herausgefiltert werden.

Dazu schl�gt Lowe den sogenannten "ratio test" vor. 
Dabei wird die Distanz des erst- und zweitbesten Matchs verglichen. Wenn das Verh�ltnis zwischen den beiden Distanzen kleiner als 0.8 ist, wird der erstbeste Match behalten. 
Ansonsten wird der Match verworfen.
Dieses Vorgehen basiert auf der Annahme, dass die Distanz beim erstbesten Match deutlich niedriger als die Distanz beim zweitbesten Match sein muss, damit der erstbeste Match als zuverl�ssig gilt. 
Sind die Distanzen zu �hnlich, kann kein eindeutiger Match f�r den Deskriptor gefunden werden. \cite[S.104]{sift2004}


