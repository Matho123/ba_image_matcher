\section{Test-Metriken}
\label{sec:TestMetriken}
Das Bilderkennungssystem soll die Bilder in zwei Klassen einordnet: Duplikate und Unikate. 
Da es in diesem Fall bei der Klassifizierung nur zwei m�gliche Klassen gibt, kann man das Bilderkennungssystem als bin�ren Klassifikator bezeichnen. 

Die Performance von bin�ren Klassifikatoren kann durch die Werte einer Wahrheitsmatrix quantifiziert werden. 
Die Wahrheitsmatrix gibt dabei an, wie viele richtige und falsche Entscheidungen das System bei der Klassifikation getroffen hat. \cite[S. 170]{classification}
\begin{table}[h]
\begin{tabular}[h]{ l | l | l}
&richtig&falsch \\
System Klass.:&&\\
\hline
Duplikat&Anzahl richtige Duplikate (TP) &Anzahl falsche Duplikate (FP) \\
\hline
Unikat&Anzahl falsche Unikate (FN) &Anzahl richtige Unikate (TN)
\end{tabular}
\caption{Wahrheitsmatrix f�r Duplikaten-Suche}
\label{Wahrheitsmatrix f�r Duplikaten-Suche}
\end{table}

Aus den Werten der Wahrheitsmatrix lassen sich weitere Metriken ableiten, die f�r die Bewertung eines bin�ren Klassifikators n�tzlich sein k�nnen. Interessant f�r diese Arbeit sind Recall, Spezifizit�t und Balancierte-Genauigkeit.

Der Recall, oder auch true positive rate, gibt das Verh�ltnis zwischen den Duplikaten, die das System korrekt klassifiziert hat, und allen Duplikaten, die sich in dem Suchdatensatz befinden, an. \cite[S. 172]{classification}

$$Recall = \frac{ TP }{ TP + FN }$$

Die Spezifizit�t, oder auch true negative rate, gibt das Verh�ltnis zwischen den Unikaten, die das System korrekt klassifiziert hat, und allen Unikaten, die sich im Suchdatensatz befinden an. \cite[S. 172]{classification}

$$Specificity = \frac{ TN }{ FP + TN }$$

Die Accuracy (dt. Genauigkeit), gibt das Verh�ltnis zwischen den Bildern, die das System korrekt klassifiziert hat, und allen Bildern im Suchdatensatz an.
Sie spiegelt die F�higkeit des Systems Bilder richtig zu Klassifizieren wieder. \cite[S. 171]{classification}

$$Accuracy = \frac{ TP + TN }{ TP + TN + FP + FN } = \frac{ Recall * P + Specificity * N }{ P + N }$$

Allerdings ist die Genauigkeit kein zuverl�ssiger Wert, wenn man mit unausgeglichenen Datens�tzen arbeitet. 
Ein Datensatz gilt dann als unausgeglichen, wenn einer Klassifikation mehr Elemente angeh�ren, als der anderen Klassifikation. \cite[S. 171]{classification}
In den Testdatens�tzen, die f�r diese Arbeit verwendet werden, befinden sich mehr Unikate als Duplikate.
Dieses Ungleichgewicht kann die Genauigkeit stark beeinflussen. So kann zum Beispiel eine gute Spezifizit�t einen schlechten Recall ausgleichen, wenn der Datensatz zum gr��ten Teil aus Unikaten besteht.

Daher kommt bei unbalancierten Datens�tzen die balanced-accuracy (dt. Balancierte-Genauigkeit) zum Einsatz. \cite[S. 175]{classification} 

$$Balanced-Accuracy = \frac{ Recall + Specificity }{ 2 }$$

Da sie den Durchschnitt aus Recall und Spezifizit�t bildet, werden Unterschiede zwischen den beiden Werten ausgeglichen. Dadurch ist die Balancierte-Genauigkeit robust gegen�ber unausgeglichenen Datens�tzen.
