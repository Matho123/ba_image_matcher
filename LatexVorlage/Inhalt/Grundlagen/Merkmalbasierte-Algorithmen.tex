\section{Merkmalbasierte Algorithmen}
\label{sec:m-based}
Ziel der "feature-based" (dt. Merkmalbasierten) Algorithmen ist es markante Punkte innerhalb von Bildern zu finden und zu beschreiben. Die Algorithmen bestehen dabei im Grunde aus zwei Schritten:
\begin{enumerate}
\item Schl�sselpunktsuche, bei der nach Koordinaten innerhalb eines Bildes gesucht wird, an denen sich markante Punkte befinden.
\item Erstellung von Deskriptoren, die den Bereich um die Schl�sselpunkte beschreiben. Diese sollen sp�ter mit Deskriptoren aus anderen Bildern verglichen werden, um gemeinsame Merkmale zu finden. 
\end{enumerate}

Wie genau diese beiden Schritte implementiert sind ist je nach Algorithmus unterschiedlich.

Dabei soll sichergestellt werde, dass die gefundenen Merkmale robust gegen�ber Rotation, Verschiebung und Skalierung sind. Auch der Einfluss durch Bildrauschen und Verzerrung soll m�glichst gering sein. \cite[S. 2]{sift1999}

H�ufig handelt es sich bei den gefundenen Merkmalen um Rand- und Eckpunkte oder Details auf einer Fl�che.