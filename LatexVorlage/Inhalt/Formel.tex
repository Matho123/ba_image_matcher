\chapter{Formel zur Bestimmung der Bild�hnlichkeit}
\label{sec:Formel}
In diesem Kapitel soll eine Formel definiert werden, mit der ein �hnlichkeits-Score errechnet wird. 
Mit diesem Score soll eine Aussage dar�ber getroffen werden, wie �hnlich sich zwei Bilder sind.
Als Eingabe erh�lt die Formel Matches, die durch den Brute-Force Matcher (Abschnitt ~\ref{sec:BFM}) gefunden und durch Lowes ratio test (Abschnitt ~\ref{sec:RatioTest}) gefiltert wurden.

\section{Vor�berlegung}
\label{FormelGedanken}
Je niedriger die Distanz eines Matches desto �hnlicher sind sich die in dem Match verglichenen Deskriptoren. (Abschnitt ~\ref{sec:DeskriptorenVergleichen})
Und je �hnlicher die Deskriptoren desto �hnlicher sind sich die verglichenen Bilder. 
Daher sollte die durchschnittliche Distanz aller Matches Aufschluss dar�ber geben, wie �hnlich sich zwei Bilder sind.

Die Werte des �hnlichkeits-Score sollten au�erdem auf einen festgelegten Bereich abgebildet werden.
Dadurch ist der �hnlichkeits-Score einer Bildpaarung gut mit dem �hnlichkeits-Score einer anderen Bildpaarung vergleichbar.

\section{Berechnung}
$$score = 1 - \frac{\sum_{d \in D} \frac{d}{max(D)}}{| D |} $$

Es werden alle Distanzen aller gefilterten Matches normalisiert und aufsummiert.
Um die Werte auf den Bereich zwischen 0 und 1 abzubilden, wird die Distanz durch die Maximaldistanz geteilt.
Bei der Normalisierung sind Werte kleiner Null sind nicht m�glich, da alle Distanzen mindestens 0 betragen.

Urspr�nglich war f�r die Formel die herk�mmliche Min-Max-Normalisierung mit $\frac{d - min(D)}{max(D)-min(D)}$ vorgesehen. 
Allerdings hat sich diese bei weiterer �berlegung als ungeeignet herausgestellt. 
F�r hohe Werte bei $min(D)$, die f�r gro�e unterscheide in den verglichenen Bildern sprechen, kommen bei der Min-Max-Normalisierung niedrige Werte heraus, wenn $min(D)$ ausreichend nah an $max(D)$ liegt. 
Dadurch kann es in solchen F�llen zu false-positives, also Bildern die f�lschlicherweise als Duplikat erkannt werden, kommen. Durch das auslassen von $min(D)$ in der Formel werden die Distanz-Werte nicht durch ein hohes $min(D)$ beschnitten.

Die Summe der normalisierten Werte wird durch die Anzahl der gefilterten Matches geteilt, um einen normalisierten Durchschnittswert zu erhalten.
Der �hnlichkeits-Score ergibt sich aus der Differenz von 1 und dem normalisierten Durchschnittswert und liegt im Bereich zwischen 0 und 1. Je h�her der Score desto �hnlicher sind sich die verglichenen Bilder. 
Ein Score von 1 w�rde bedeuten, dass beide Bilder identisch sind.

Der Mindest-Score, der erreicht werden muss, damit ein Bild als Duplikat gilt, soll im Verlauf der Tests ermittelt werden.