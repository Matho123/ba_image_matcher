\section{Vor�berlegung}
\label{sec:BSGedanken}
Die Auswertungen der Tests haben gezeigt, das die Deskriptoren einen sehr hohen Speicherbedarf haben und sich daher nicht f�r die Persistierung in einer Datenbank eignen.
Auch die Laufzeit von feature-based Algorithmen ist deutlich h�her als die Laufzeit des pHash Algorithmus, wenn alle Deskriptoren in der Datenbank verglichen werden sollen.

Au�erdem w�rde der Umstieg vom momentan genutzten pHash-System auf ein feature-
based-System bedeuten, das die bereits durch Spreadshirt gesammelten Hash-Werte nicht
mehr brauchbar sind. F�r das neue System m�ssten f�r mehrere Millionen abgelehnte
Designs die Deskriptoren berechnet und gespeichert werden, was ein gro�er, wenn auch
einmaliger, Aufwand ist.

Daher wird f�r das neue Bilderkennungs-System ein Ansatz vorgesehen, der sich die niedrigen Speicherkosten und den schnellen Vergleich von Bildern mit pHash zunutze macht. 
Gleichzeitig soll es aber auch von der Robustheit der feature-based Algorithmen profitieren.

Die Idee ist, dass pHash genutzt wird, um aus der Datenbank eine Vorauswahl mit potenziellen Duplikaten zu treffen.
Danach wird ein feature-based Algorithmus genutzt, um den besten Match zu finden. 
Die Schl�sselpunkte und Deskriptoren des Suchbildes und der Kandidaten aus der Vorauswahl werden dabei zur Laufzeit extrahiert.

Mit diesem Hybrid-Ansatz k�nnen die bereits bei Spreadshirt gesammelten Hash-Werte weiterhin genutzt werden. 
Daher muss das bisherige pHash-System nicht ersetzt werden und wird lediglich erweitert. 
Au�erdem bleiben auch die Persistierungskosten f�r die abgelehnten Designs gleich, da weiterhin nur ein einzelner 64 Bit pHash-Wert pro Bild gespeichert werden muss.

Die Zuverl�ssigkeit des Systems wird aber dennoch weiterhin stark durch die Zuverl�ssigkeit des pHash-Algorithmus beschr�nkt. 
Der feature-based Algorithmus hat in diesem Entwurf nur Zugriff auf die Bilder, die in der Vorauswahl durch den pHash Algorithmus gesammelt wurden. Daher k�nnen nur die Duplikate erkannt werden, die auch durch pHash erkannt wurden. 

Die Tests haben gezeigt, das der pHash Algorithmus Schwierigkeiten hat Duplikate zu erkennen die gespiegelt oder rotiert sind. Besonders problematisch sind aber auch Duplikate, bei denen das Originalbild als Teil des Motivs auftauchen.
Daher wird es n�tig sein das System f�r die Vorauswahl so anzupassen, sodass pHash mit den diesen F�llen umgehen kann. 
Das Ziel ist f�r die Vorauswahl einen m�glichst hohen Recall zu erreichen, damit m�glichst alle potenziellen Duplikate durch den feature-based Algorithmus gepr�ft werden k�nnen.


